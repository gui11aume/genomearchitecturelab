\documentclass[12pt]{article}
\usepackage{url}
\usepackage[noanswer]{exercise}

\setlength{\ExerciseSkipBefore}{1em}
\setlength{\ExerciseSkipAfter}{1em}
\renewcommand{\ExerciseHeader}{\noindent\textbf{\normalsize\ExerciseName\space\ExerciseHeaderNB\quad}\par}
\renewcommand{\AnswerHeader}{\noindent\textbf{Answer to \normalsize\ExerciseName\space\ExerciseHeaderNB\quad}\par}

\title{Stats exam 2014}
\author{Guillaume Filion}

\begin{document}

\maketitle

Please submit your answers by email to
\texttt{guillaume.filion@gmail.com} before the deadline. Each
question is worth 1 point. Every fraction of 24 hours passed
the deadline will be penalized by 2 points. You are encouraged
to work in group on these exercises.
You can submit a joint answer sheet and will receive the same
grade as your team members. In this case, indicate the name of
all the students participating to the work.

The answers will be posted online at the address below. \par
\noindent \url{www.genomearchitecture.com/static/misc/statsexam_answers_2014.pdf}

\section{Course questions}

\begin{Exercise}[label={exo1}]
  What are the main steps of a test?
\end{Exercise}
\begin{Answer}[ref={exo1}]
  \begin{enumerate}
    \item State the null hypothesis
    \item Choose a statistic
    \item Compute the null distribution
    \item Compare the observed value of the stastic to the null
  \end{enumerate}
\end{Answer}

\begin{Exercise}[label={exo2}]
  What is the p-value of a test?
\end{Exercise}
\begin{Answer}[ref={exo2}]
  The probability that the statistic is more extreme, given that
  the null hypothesis is true.
\end{Answer}

\begin{Exercise}[label={exo3}]
  What is the power of a test?
\end{Exercise}
\begin{Answer}[ref={exo3}]
  The probability of rejecting the null hypothesis.
\end{Answer}

\begin{Exercise}[label={exo4}]
  How can you increase the power of a test?
\end{Exercise}
\begin{Answer}[ref={exo4}]
  By increasing the level or by increasing the sample size.
\end{Answer}

\begin{Exercise}[label={exo5}]
  What is the statistic of the $t$-test?
\end{Exercise}
\begin{Answer}[ref={exo5}]
  The effect size.
\end{Answer}

\section{Problem}

  An insecure implementation of a password-protected acess
  is to compare letter by letter the text enterred by the
  user to the real password, and to deny access as soon
  as a different letter is found. This is insecure because
  it opens a possibility for \textit{a timing attack}, which
  is an attempt to hack a password by measuring the time it
  takes for the computer to respond.

  \url{http://en.wikipedia.org/wiki/Timing_attack}

  Suppose that my real password is \texttt{kotiki125}. If
  the hacker enters \texttt{abcdefg}, the insecure method will
  deny access immediately after comparing the first letter
  because \texttt{a} is different from  \texttt{k}. But
  if the hacker enters \texttt{kotiki123}, this method
  will deny access upon comparing the 9-th letter, which will
  take roughly 9 times as long. With this information, the
  hacker could deduce that the first 8 letters of
  \texttt{kotiki123} are correct.

  The idea of a timing attack is to try all the letters
  one by one and measure the time it takes for the computer
  to respond. Every time a letter matches the password, the
  response time will be slightly slower. By repeating the
  process, the password can be decrypted completely.

\begin{Exercise}
  Assume that there are 64 valid password letters
  (small letters, capital letters, numbers, \texttt{\_} and
  \texttt{\~}, and that you want to decrypt a password
  generated at random. You enter a text. What is the
  probability that the first letter matches the password?
  What is the probability that the first 3 letters match
  the password?
\end{Exercise}

  Based on previous measurements, you know that the
  time to compare two letters is 10 ns. In addition, for
  every letter comparison, there is a random overhead due
  to other processes running on the server. This overhead
  has an exponential distribution with rate 0.125 ns$^{-1}$.

\begin{Exercise}
  The function \texttt{rexp(8, rate=.125)} in \texttt{R}
  generates a random sample of size 10 drawn from an
  exponential distribution with rate 0.125 ns$^{-1}$.
  How to generate a random sample of size 8 representing
  the response time of the server when the first letter of
  the text does not match the password?
\end{Exercise}

  You have entered the text \texttt{aaaaaaaa} 8 times
  and have observed the following response times (in ns)
  23.0, 15.7, 21.4, 12.8, 13.3, 21.3, 38.2, 15.4.

\begin{Exercise}
  Do you think that the first letter of the password is
  \texttt{a}? Quantify your certainty. Before you set up 
  a complicated statistical test, look at the numbers
  carefully.
\end{Exercise}

  You know that the password starts with \texttt{wy6qU}.
  You now enter \texttt{wy6qUaaaaa} 8 times and obtain
  the following response times (in ns)
  122.0, 83.8, 71.4, 136.6, 93.6, 88.1, 84.1, 123.0

  In \texttt{R} you can compute the sum of 6 exponential
  random variables with rate 0.125 ns$^{-1}$ by using
  \texttt{sum(rexp(6, rate=.125)}. Similarly, you can
  generate a sample of size 8 representing the response
  time of the server if the 6-th letter does not match
  by using \texttt{replicate(8,60+sum(rexp(6,rate=1/8)))}.

\begin{Exercise}
  Using this information, design a statistical test to
  know whether the 6-th letter of the password is \texttt{a}.
  Quantify your certainty.
\end{Exercise}

\begin{Exercise}
  What is the power of your test?
\end{Exercise}

\end{document}
